\documentclass[11pt]{article}

% font things
\usepackage{amsmath}
\usepackage{MnSymbol} % math things
% serif: if MinionPro doesn't work, use mathptmx
\usepackage[lf, mathtabular, minionint]{MinionPro} % Minion
% \usepackage{mathptmx}   % times
% sans font: use roboto if MyriadPro doesn't work
% \usepackage{MyriadPro} 
\usepackage{roboto}     % sans 


% begin old preamble

% --- character encoding ---
% \usepackage[latin1]{inputenc}
% \usepackage[T1]{fontenc}


\usepackage[top=.8in, bottom=.8in, left=1in, right=1in]{geometry}

% --- old font ---
% \renewcommand{\rmdefault}{pplx}
% \usepackage[sc]{mathpazo}
% \usepackage[OT1, euler-hat-accent]{eulervm}
\usepackage[usenames, dvipsnames, svgnames]{xcolor}
\usepackage{enumitem}

% --- styling ---
\usepackage{titling}
\usepackage[small, compact]{titlesec}
\setitemize[0]{leftmargin=*}
\usepackage{multicol, multirow}
\usepackage{epsfig, subfigure, subfloat, graphicx}
\usepackage{anysize, indentfirst, setspace}
\usepackage{verbatim, rotating, xfrac}
\usepackage{gensymb}
\usepackage{caption, hanging}
\newcommand{\mc}[1]{\multicolumn{1}{c}{#1}}
%\parindent 0pt
%\setdefaultenum{a.}{i.}{A}{1}
%\setdefaultitem{}{\textperiodcentered}{}{}
\usepackage{booktabs}
\usepackage{dcolumn}
\usepackage{caption, hanging}
\usepackage{tikz}
\usetikzlibrary{shapes,arrows,backgrounds}
%\setdefaultenum{a.}{1)}{i.}{a.}
\parindent 0pt

\makeatletter
\newcommand{\distas}[1]{\mathbin{\overset{#1}{\kern\z@\sim}}}%
\newsavebox{\mybox}\newsavebox{\mysim}
\newcommand{\distras}[1]{%
  \savebox{\mybox}{\hbox{\kern3pt$\scriptstyle#1$\kern3pt}}%
  \savebox{\mysim}{\hbox{$\sim$}}%
  \mathbin{\overset{#1}{\kern\z@\resizebox{\wd\mybox}{\ht\mysim}{$\sim$}}}%
}
\makeatother








\begin{document}

\title{\Large{\bf{\vspace{-100pt}Mathematics for Political Science \vspace{-15pt}}}}
\author{\large{Lesson 3: Calculus}}
\date{\vspace{-5pt}\large{Exercises \vspace{-10pt}}}

\maketitle

\hrule


\begin{enumerate}


\item (Gill 5.1 [adapted]) Find the following finite limits:
\begin{enumerate}
\item $\displaystyle\lim_{x\rightarrow 4} [x^2 - 6x + 4]$ %-4
\item $\displaystyle\lim_{x\rightarrow 0} [\frac{x-25}{x+5}]$ %-5
\item $\displaystyle\lim_{x\rightarrow 4} [\frac{x^2}{3x-2}]$ %1.6
\item $\displaystyle\lim_{x\rightarrow 1} [\frac{x^2 - 1}{x-1}]$ %2
\end{enumerate}



\item (Gill 5.3 [adapted]) Find the following infinite limits and graph:
\begin{enumerate}
\item $\displaystyle\lim_{x\rightarrow \infty} [\frac{9x^2}{x^2 + 3}]$ %9
\item $\displaystyle\lim_{x\rightarrow \infty} [\frac{3x-4}{x+3}]$ %3
\item $\displaystyle\lim_{x\rightarrow \infty} [\frac{2^x - 3}{2^x + 1}]$ %1
\end{enumerate}



\item (Gill 5.5 [adapted]) Calculate the following derivatives:
\begin{enumerate}
 \item $\frac{d}{dx} 3x^{\frac{1}{3}}$
 \item $\frac{d}{dt}(14t - 7)$
 \item $\frac{d}{dy}(y^3 + 3y^2 - 12)$ 
 \item $\frac{d}{dx}(x^2 + 1)(x^3 - 1)$
 \item $\frac{d}{dy}(y^3 - 7)(1 + \frac{1}{y^2})$
 \item $\frac{d}{dy}(y - y^{-1})(y - y^{-2})$
 \item $\frac{d}{dx}\frac{4x - 12x^2}{x^3-4x^2}$
 \item $\frac{d}{dy}e^{y^2 - 3y + 2}$
 \item $\frac{d}{dx}\ln (2\pi x^2)$
\end{enumerate}


\item Consider the function $k(x) = 2(8(x^4 + 2) - 1)^2$.  Find the derivative by:
\begin{enumerate}
\item Expanding the polynomial and calculating the derivative using the power rule.
\item Expressing $k(x)$ as the result of three nested functions $f(g(h(x)))$ and applying the chain rule.
\end{enumerate}
~~~~Show that these approaches yield the same answer.
%4(8(x^4+2)-1)*8*4x^3




\item For each of the functions:
\begin{align*}
f(x) &= 3x^2 - 7x + 2 &  g(x) &= 8x^3 - 46x^2 + 73x - 35
\end{align*}
% f'(x) = 6x - 7, x =7/6.... g'x=24x^2 - 92x + 72
\begin{enumerate}
\item Sketch a plot the function on the interval $[0,5]$ (calculate f(x) for integer values of x to get a general idea of the shape of the function).
\item Identify the values of x that generate local maxima or minima (ignoring endpoints).
\item Show mathematically whether these are maxima or minima.
\end{enumerate}



\item Find the value of $x$ that maximizes the function $\ell(x) = 2\ln(x) - x - \ln(2x+1)$ using the following approach.
\begin{enumerate}
\item Take the derivative of $\ell(x)$ and set it equal to 0.
\item Manipulate the expression to remove fractions and express it as a quadratic.
\item Solve for $x$.
\end{enumerate}
% approx .78, -1.28



\item Find the partial derivatives of the function $(eR(\frac{f}{f+g}))^h$ with respect to $e$ and $f$.




\item (Gill 5.13 [adapted]) Calculate the following indefinite integrals:
\begin{enumerate}
 \item $\int 4y^3 dy$ 
 \item $\int (x^2 - x^{-\frac{1}{2}}) dx$ 
 \item $\int 360t^6 dt$
\end{enumerate}



\item (Gill 5.10 [adapted]) Solve the following definite integrals using the antiderivative method:
\begin{enumerate}
 \item $\int_{6}^{8} x^3 dx$ 
 \item $\int_{1}^{9} 2y^5 dy$ 
 \item $\int_{-1}^{0} (3x^2 - 1)dx$ 
 \item $\int_{-1}^{1} (14 + x^2) dx$ 
 \item $\int_{2}^{4} e^y dy$ 
 \item $\int_{2}^{4} \sqrt{t}dt$
\end{enumerate}




\item (Gill 5.11) Calculate the area of the following function that lies above the x-axis and over the domain $[-10,10]$:
\begin{equation*}
f(x) = 4x^2 + 12x - 18
\end{equation*}





\end{enumerate}



\end{document}