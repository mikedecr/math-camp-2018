\documentclass[11pt]{article}

% font things
\usepackage{amsmath}
\usepackage{MnSymbol} % math things
% serif: if MinionPro doesn't work, use mathptmx
\usepackage[lf, mathtabular, minionint]{MinionPro} % Minion
% \usepackage{mathptmx}   % times
% sans font: use roboto if MyriadPro doesn't work
% \usepackage{MyriadPro} 
\usepackage{roboto}     % sans 


% begin old preamble

% --- character encoding ---
% \usepackage[latin1]{inputenc}
% \usepackage[T1]{fontenc}


\usepackage[top=.8in, bottom=.8in, left=1in, right=1in]{geometry}

% --- old font ---
% \renewcommand{\rmdefault}{pplx}
% \usepackage[sc]{mathpazo}
% \usepackage[OT1, euler-hat-accent]{eulervm}
\usepackage[usenames, dvipsnames, svgnames]{xcolor}
\usepackage{enumitem}

% --- styling ---
\usepackage{titling}
\usepackage[small, compact]{titlesec}
\setitemize[0]{leftmargin=*}
\usepackage{multicol, multirow}
\usepackage{epsfig, subfigure, subfloat, graphicx}
\usepackage{anysize, indentfirst, setspace}
\usepackage{verbatim, rotating, xfrac}
\usepackage{gensymb}
\usepackage{caption, hanging}
\newcommand{\mc}[1]{\multicolumn{1}{c}{#1}}
%\parindent 0pt
%\setdefaultenum{a.}{i.}{A}{1}
%\setdefaultitem{}{\textperiodcentered}{}{}
\usepackage{booktabs}
\usepackage{dcolumn}
\usepackage{caption, hanging}
\usepackage{tikz}
\usetikzlibrary{shapes,arrows,backgrounds}
%\setdefaultenum{a.}{1)}{i.}{a.}
\parindent 0pt

\makeatletter
\newcommand{\distas}[1]{\mathbin{\overset{#1}{\kern\z@\sim}}}%
\newsavebox{\mybox}\newsavebox{\mysim}
\newcommand{\distras}[1]{%
  \savebox{\mybox}{\hbox{\kern3pt$\scriptstyle#1$\kern3pt}}%
  \savebox{\mysim}{\hbox{$\sim$}}%
  \mathbin{\overset{#1}{\kern\z@\resizebox{\wd\mybox}{\ht\mysim}{$\sim$}}}%
}
\makeatother





\title{\Large{\bf{\vspace{-100pt}Mathematics for Political Science \vspace{-15pt}}}}
\author{\large{Lesson 2: Algebra}}
\date{\vspace{-5pt}\large{Exercises \vspace{-10pt}}}
\begin{document}
\maketitle

\hrule



\begin{enumerate}

\item Solve the following equations for x:
\begin{enumerate}
\item $12x + 2 = 18x$ %x = 1/3
\item $-6 - 4x = -3 - 8x$  %x = 3/4
\end{enumerate}


\item Express $\alpha$ in terms of the other unknown variables:
\begin{enumerate}
\item $3\alpha - 8\theta = \alpha + 2\beta$ %alpha = beta + 4 theta
\item $\alpha x + \alpha y = \alpha x^2 + \alpha y^2 + 4$ %alpha = 4/(x + y - x^2 - y^2)
\end{enumerate}


\item (Gill 1.6) Solve the following inequalities so that the variable is the only term on the left-hand side:
\begin{enumerate}
\item $x - 3 < 2x + 15$  %x > -18
\item $11 - \frac{4}{3}t > 3$  % t < 6
\item $\frac{5}{6}y + 3(y-1) \leq \frac{11}{6}(1-y) + 2y$  % y \leq 29/22
\end{enumerate}


\item Find the values of $x$ where $f(x)=0$ using factorization:
\begin{enumerate}
\item $x^2 + 5x - 14$ % x=2, x=-7 (x-2)(x+7)
\item $x^2 - 8x + 16$ % x=4 (x-4)^2
\item $3x^2 + 9x - 30$ % x=2, x=-5 (3x-6)(x+5)
\end{enumerate}


\item Solve the following equations for $x$ using the quadratic formula:
\begin{enumerate}
\item $18x^2 + 10x = 3 - 15x$ %x=1/9, x=-1.5
\item $20x^2 + 2x - 3 = 5 + 20x - 15x^2$ % x= -2/7, x=4/5
\end{enumerate}


\item Solve the following systems of equations for $a$ and $b$ using the ``direct substitution'' approach:
\begin{enumerate}
\item $b + 5a = 2$ \\ $7b - 6a = 14$ %a=0, b=2
\item $3(a+b) + 7a = 8(b-1) + 33$ \\ $-3a + 4(1-b) = 4(1-a) - 15$ %a=5, b=5
\end{enumerate}



\item Solve the following systems of equations for $c$ and $d$ using the ``elimination'' approach:
\begin{enumerate}
\item $3c + 4d = 13$ \\ $2c + 5d = 4$ %c=7, d=-2
\item $c + 4d + 36 = 10d - 3c$ \\ $2(c+1) + 2(d+1) = 6$ %c=-3, d=4
\end{enumerate}


\item Solve this system of equations for $x$ and $y$ in terms of $\alpha$: \\
$~~~~~~~~2x + y = 10\alpha + 5$ \\
$~~~~~~~~3x + 3y = 18\alpha + 9$ % x=4a + 2, y=2a + 1


\item Solve this system of equations for q, r, and s: \\
$~~~~~~~~2q + 4r + s = 1$ \\
$~~~~~~~~4(q+1) + 7(1-r) = 2s + 16$ \\
$~~~~~~~~8q + 4r - 2s = 5q + 19r + 4s$ % q=1, r=-1, s=3


\item Calculate the dot product of the vectors below.
\begin{enumerate}
\item $[3, 4, 1, 7, 0] \cdot [5, 2, 2, 0, 3]$ %25
\item $[4, 1, 3] \cdot [0, 7, 5]$ %22
\end{enumerate}



\item (Gill 3.9) For the following matrix, calculate $\textbf{X}^n$ for $n = 2, 3, 4, 5$.  Write a rule for calculating higher values of $n$.
\[
\left[\begin{array}{ccc}
0 & 0 & 1 \\
0 & 1 & 0 \\
1 & 0 & 0 \\
\end{array}\right]
\]
% odd powers look like this, even powers are the identity matrix


\item Using the matrix below, show the identities of multiplication and addition for matrices:
\[
\left[\begin{array}{cc}
a & b \\
c & d \\
\end{array}\right]
\]



\item Perform the following matrix multiplications, or explain why they are not possible:
\begin{enumerate}
\item $\left[\begin{array}{cccc}
4 & 5 & 5 & 2 \\
\end{array}\right]
\left[\begin{array}{cc}
1 & 3 \\
8 & 1 \\
0 & 9 \\
6 & 4 \\
\end{array}\right]$
% [56  70]
\item $\left[\begin{array}{ccc}
a & b & c \\
d & e & f \\
g & h & i \\
\end{array}\right]
\left[\begin{array}{c}
p \\
q \\
r \\
\end{array}\right]$
\item $\left[\begin{array}{ccc}
\alpha & \beta    & \gamma \\
\delta & \epsilon & \eta \\
\end{array}\right]
\left[\begin{array}{cc}
\lambda & \sigma \\
\end{array}\right]$
% not possible - doesn't conform
\end{enumerate}




\item Multiply the matrices below to show that order matters for matrix multiplication: 

\begin{tabular}{cccc}
&
a. $\left[\begin{array}{ccc}
4 & 7 & 1 \\
\end{array}\right]
\left[\begin{array}{c}
3 \\
0 \\
5 \\
\end{array}\right]$
&
~~~~~~~~
&
$\left[\begin{array}{c}
3 \\
0 \\
5 \\
\end{array}\right]
\left[\begin{array}{ccc}
4 & 7 & 1 \\
\end{array}\right]$
\end{tabular}


\begin{tabular}{cccc}
&
b. $\left[\begin{array}{cc}
4 & 8 \\
1 & 6 \\
2 & 2 \\
\end{array}\right]
\left[\begin{array}{ccc}
9 & 6 & 3 \\
1 & 5 & 3 \\
\end{array}\right]$
&
~~~~~~~~
&
$\left[\begin{array}{ccc}
9 & 6 & 3 \\
1 & 5 & 3 \\
\end{array}\right]
\left[\begin{array}{cc}
4 & 8 \\
1 & 6 \\
2 & 2 \\
\end{array}\right]$
\end{tabular}




\end{enumerate}



\vfill
\begin{center}
\small{Thanks to Dave Ohls, Brad Jones, and Sarah Bouchat for past years' materials}
\end{center}

\end{document} 